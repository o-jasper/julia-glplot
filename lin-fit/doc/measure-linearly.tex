\documentclass{article}

\author{Jasper den Ouden}

\newcommand{\half}{\frac{1}{2}}

\begin{document}

\section*{Most likelyhood on linear additions}
The method of most likelyhood assumes gaussian errors, this leads to a
 method of least squares by taking the logarithm:
$$\chi^2 \equiv log(\mathrm{Probability})= 
  log(\prod_kexp((y(x_k)-y_k)^2/2\sigma^2_k)=
  \half\sum_k\sigma_k^{-2}(y(x_k)-y_k)^2$$
Where $k$ is a number for the independent measurements.

For the model of linear additions:
$$y(x)= \sum_i a_if_i(x)$$
We have:
$$\chi^2=\half\sum_k\sigma_k^{-2}(\sum_{i}a_if_i(x_k)-y_k)^2$$
leading to linear equations when minimizing:
$$\partial_{a_j}\chi^2= \sum_k\sigma_k^{-2}(y(x_k)-y_k)\cdot f_j(x_k)=0
\sum_k\sigma_k^{-2}(\sum_ia_if_i(x_k)-y_k)\cdot f_j(x_k)=0
 \Rightarrow$$
$$\sum_i a_i \sum_k\sigma_k^{-2}f_i(x_k)\cdot f_j(x_k)=
   \sum_{ik}\sigma_k^{-2}y_k\cdot f_j(x_k)$$

Now note that each of the sums here can be added together at any point. We 
dont need to store a list of stuff and then add those together.

\subsection*{Subsets of this}
Clearly stuff like $y=Ax+b = \sum A_i\cdot x_i + b$ is a subset, with
$f_i(x)=x_i$ and $f_{N+1}(x)=1$. Of course $y=Mx$ is also  subset.

\subsection*{Fitting $y=\sum_ia_if_i(x)$ using fitting $y=Mx$}
Conversely, we can use $y=Mx$ to fit $y=\sum_ia_if_i(x)$, by 
$x_i\rightarrow f_i(x)$ (different x on sides arrow) Note that the errors are
those on $y$ so those aren't 'translated'. All that is really happening here 
is that we're providing the functions already calculated here, though.

\section*{Useful for}
Many problems can be linearized in approximations. And linear problems
as $L\Phi=s$, with $L$ a linear operator has Green functions, so that
$LG=\delta$ and then
 $L\int G(y-x)s(y) dy = \int \delta(y-x) s(y) dy= s(x)$ so 
$\Phi=\int G(y-x)s(y)dy$ is a solution. This can also be divided into volumes
$V_i$ so that for $i/ne j$, $V_i\cap V_j= \phi$ total volume $V=\cap_iV_i$,
then $\Phi= \int_V G(y-x)s(y)dy = \sum_i \int_{V_i}G(y-x)s(y)dy$ which can
help making assumptions about the source.

Of course you probably have already done it. If you have added forces from 
different sources of gravity, electrostatic forces etcetera. Adding the 
gravity of a sphere is basically using a $V_i=\mathrm{That sphere}$ and 
assuming the source everywhere else zero.
\footnote{Maybe this is No 1. of the overcomplicatedly teached very simple 
things.}

\subsection*{Linear electrical circuits}
The first thing to treat is electrical circuits, where the source is a 
'magical' current input.\footnote{'Heat conduction circuits' are analogous, 
but just don't have inductance.} Essentially, it comes down to the same as the
previous.

\end{document}
